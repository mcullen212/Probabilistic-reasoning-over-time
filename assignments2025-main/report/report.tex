\documentclass{article}
\usepackage{amsmath, amssymb, amsfonts}
\usepackage[utf8]{inputenc}
\usepackage[english]{babel}
\usepackage{fontenc}
\usepackage{graphicx}
\usepackage{geometry}
\usepackage{fancyhdr}
\usepackage{titlesec}
\usepackage{microtype}
\usepackage{float}
\usepackage{hyperref}
\usepackage{cleveref}
\usepackage{placeins}

\geometry{a4paper, margin=1in}

\pagestyle{fancy}
\fancyhf{}
\fancyhead[L]{\includegraphics[height=1cm]{lth.png}}
\fancyhead[C]{Campoli, Cullen de Erquiaga, Ghigliani, Mutz}
\fancyhead[R]{\today}
\fancyfoot[C]{\thepage}
\fancyfoot[L]{Simulation Tools - Project 1}
\fancyfoot[R]{Exchange Students}

\titleformat{\section}
  {\Large\bfseries\sffamily}
  {\thesection.}{0.5em}{}
\titlespacing*{\section}{0pt}{1.5ex plus 1ex minus .2ex}{1ex plus .2ex}

\begin{document}

\begin{titlepage}
    \centering
    \vspace*{1.5cm}
    {\LARGE\bfseries Lund University\par}
    \vspace*{0.5cm}
    {\LARGE\bfseries Simulation Tools\par}
    \vspace{1cm}
    {\Large Project 1\par}
    \vspace{1cm}
    {\Large Campoli Lucas\par}
    \vspace{0.5cm}
    {\Large Cullen de Erquiaga Magdalena Itziar\par}
    \vspace{0.5cm}
    {\Large Ghigliani Franco\par}
    \vspace{0.5cm}
    {\Large Mutz Matias\par}
    \vspace{1cm}
    {\large Exchange Students\par}
    \vspace{2cm}
    \includegraphics[width=0.7\textwidth]{lth.png}
    \vfill
    {\large \today\par}
\end{titlepage}

\tableofcontents
\newpage

\section{Task 1}

\begin{figure}[h!]
    \centering
    \includegraphics[width=0.5\textwidth]{pendulum.png}
    \caption{Elastic Pendulum}
    \label{fig:pendulum}
    \cite{pendulum}
\end{figure}

Harmonic function when the oscillation of the of the waves of the x-value have the point part tending to straight line
when the oscillations of the waves of the y-value are stable not harmonic function waves are unstable which makes
sense as the resort is affecting the plotting the higher the k constant → the more stable the waves making them tend
to a harmonic function.

\section{Task 2}

In this experiment, we implemented and tested the BDF-3 and BDF-4 methods to solve the elastic pendulum problem
described by the given system of ODEs. The right-hand side function (rhs\_function) was implemented to compute the
derivatives of the state variables, and the BDF-3 and BDF-4 methods were implemented using Newton's method (via fsolve)
as the corrector iteration.
The simulations were performed for a moderately sized spring constant k=10 and initial conditions corresponding to a
slightly stretched spring.

\subsection{\textbf{Observations}}

\begin{itemize}
    \item \textbf{BDF-3 and BDF-4 Performance:}
    \subitem - Both BDF-3 and BDF-4 methods successfully solved the elastic pendulum problem for the given parameters.
    The solutions for y1 and y2 (positions) were plotted over time, showing oscillatory behavior consistent with the
    physical interpretation of the system.
    BDF-4, being a higher-order method, provided slightly smoother and more accurate results compared to BDF-3,
    especially for longer simulation times.
    This is expected since higher-order methods generally have better accuracy
    and stability properties.
    \item \textbf{Stability and Accuracy:}
    \subitem - The BDF methods are implicit and have good stability properties for stiff problems.
    The elastic pendulum problem becomes increasingly stiff as k increases, and the BDF methods were able to handle this
    stiffness effectively.
    The use of Newton's method for the corrector iteration ensured that the implicit equations were solved accurately,
    even for larger step sizes.
    \item \textbf{Initial Steps and Lower-Order Methods:}
    \subitem - For the first few steps, where insufficient previous values were available, lower-order methods (e.g.,
    explicit Euler) were used.
    This is a common practice in multi-step methods like BDF to initialize the solution history.
    \item \textbf{Influence of Step Size:}
    \subitem - The step size h=0.01 was chosen to balance computational efficiency and accuracy. Smaller step sizes
    would improve accuracy but increase computational cost, while larger step sizes might lead to instability or loss of
    accuracy, especially for a higher k.
    \item \textbf{Comparison with Explicit Methods:}
    \subitem - Explicit methods like the explicit Euler method would struggle with this problem, especially for large k,
    due to their poor stability properties for stiff systems. The BDF methods, being implicit, are better suited for such
    problems.
    \item \textbf{Reproducibility:}
    \subitem - The experiments can be reproduced by running the provided code with the specified parameters. The results
    are consistent and can be verified by comparing the plots generated for BDF-3 and BDF-4.
\end{itemize}

\subsection{\textbf{Conclusion}}

The BDF-3 and BDF-4 methods are effective for solving the elastic pendulum problem, particularly for moderately stiff
systems.
The higher-order BDF-4 method provides better accuracy and stability compared to BDF-3, making it a preferable choice
for longer simulations or more stringent accuracy requirements.
The results align with the theoretical expectations from stability diagrams, demonstrating the practical advantages of
implicit methods for stiff ODEs.

\section{Task 3}

\subsection{\textbf{vanishing pointObserving the Behavior of Fixed Order Methods}}

Fixed-order methods, such as \textbf{Explicit Euler} and \textbf{BDF-2}, are tested against increasing values of k. The key observations include:

\begin{itemize}
    \item \textbf{Explicit Euler Method:}
    \subitem - For \textbf{small} values of k (e.g., k=1), the method provides a reasonable approximation.
    \subitem - As k \textbf{increases}, the high-frequency oscillations become more pronounced, and the explicit Euler method struggles to maintain stability.
    \subitem - For \textbf{large k}, the method produces numerical instability, leading to rapidly diverging or oscillatory results, which is a well-known issue with explicit methods for stiff problems.
    \item \textbf{BDF-2 (Implicit Method from Task 2 Example):}
    \subitem - Unlike explicit Euler, implicit methods like BDF-2 can handle stiff problems better.
    \subitem - The method remains stable for large k, though some numerical damping may be observed.
    \subitem - For moderate k, BDF-2 maintains a smooth and stable solution.
\end{itemize}

\subsection{\textbf{Comparison with Explicit Euler and Fixed-Point Iteration}}

\begin{itemize}
    \item \textbf{Explicit Euler} becomes impractical for \textbf{large k} due to stability constraints (i.e., it requires a very small time step to remain accurate).
    \item \textbf{Fixed-point iteration methods} (like the one used in BDF-2) handle large k much better but may require more computational effort.
    \item A key takeaway is that explicit Euler is \textbf{not suitable} for high-frequency oscillations, whereas implicit methods can manage them efficiently.
\end{itemize}

\subsection{\textbf{Stability Verification Using Stability Diagrams}}

The observed instability in explicit Euler aligns with stability region analysis:

\begin{itemize}
    \item \textbf{Explicit Euler has a very limited stability region}, meaning that for stiff problems (like when k is large), the numerical method exhibits instability unless the step size is extremely small.
    \item \textbf{BDF methods (implicit methods) have a much larger stability region}, allowing them to work with larger time steps while still producing accurate results.
    \item \textbf{For very large k}, even BDF-2 starts struggling, and higher-order BDF methods (e.g., BDF-4 or BDF-3) become preferable.
\end{itemize}

\subsection{\textbf{Final Takeaways}}

\begin{itemize}
    \item \textbf{Explicit Euler is not suited for large k} because of instability issues.
    \item \textbf{Fixed-point iteration methods (like BDF-2) handle stiffness well} but require more computational effort.
    \item \textbf{The stability diagrams align with our numerical results}, confirming that implicit methods are necessary for large k.
    \item \textbf{For very high k, we may need higher-order implicit methods like BDF-4 or adaptive solvers like CVODE.}
\end{itemize}

\section{Task 4}

\newpage

\bibliographystyle{IEEEtran}
\bibliography{references}

\end{document}